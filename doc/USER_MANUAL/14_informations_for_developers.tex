%------------------------------------------------------------------------------------------------%

\chapter{Information for developers of the code, and for people who want to learn how the technique works}\label{cha:developers}

%------------------------------------------------------------------------------------------------%

You can get a very simple 1D version of a demo code (there is one in Fortran and one in Python):
\begin{verbatim}
git clone --recursive https://github.com/SPECFEM/specfem1d.git
\end{verbatim}

\noindent We also have simple 3D demo source codes that implement the SEM in a single, small program, in directory\\
utils/small\_SEM\_solvers\_in\_Fortran\_and\_C\_without\_MPI\_to\_learn of the specfem3d package.
They are useful to learn how the spectral-element method works, and how to write or modify a code to implement it.
Also useful to test new ideas by modifying these simple codes to run some tests.
We also have a similar, even simpler, demo source code for the 2D case in directory\\
utils/small\_SEM\_solver\_in\_Fortran\_without\_MPI\_to\_learn of the specfem2d package.

\noindent For information on how to contribute to the code, i.e., for how to make your modifications, additions or improvements part of the
official package, see \url{https://github.com/SPECFEM/specfem3d/wiki} .

