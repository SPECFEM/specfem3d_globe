\chapter*{Copyright}\label{cha:Copyright}
\addcontentsline{toc}{chapter}{Copyright}

Main `historical' authors: Dimitri Komatitsch and Jeroen Tromp (there are now many more!).

Princeton University, USA, and CNRS / University of Marseille, France.\newline
$\copyright$ Princeton University, USA and CNRS / University of Marseille, France, April 2014\newline

\noindent
This program is free software; you can redistribute it and/or modify
it under the terms of the GNU General Public License as published
by the Free Software Foundation (see Appendix \ref{cha:License}).\newline

\noindent
Please note that by contributing to this code, the developer understands and agrees that this project and contribution
are public and fall under the open source license mentioned above.\newline

\noindent
\textbf{\underline{Evolution of the code:}}\newline

 version 8.1, November 2023:
   Daniel Peter, Julien Thurin, Lucas Sawade.
      adds EMC model support; minor fixes and code improvements; updates ADIOS2 and LIBXSMM calls.\newline

 version 8.0, March 2023:
   Wolfgang Bangerth, Stephen Beller, Ebru Bozdag, Caio Ciardelli, Congyue Cui, Armando Espindola-Carmona,
   Rene Gassmoeller, Sunny Gogar, Leopold Grinberg, Elodie Kendall, Wenjie Lei, Amanda McPherson, Ridvan Orsvuran,
   Daniel Peter, Norbert Podhorszki, Eduardo Valero Cano.
     support for new Earth, Moon \& Mars models; ADIOS2 file I/O support, GLL models for azimuthal anisotropy \& Q,
     LDDRK on GPU support, Laplacian smoothing, monochromatic source time functions, sponge absorbing boundaries,
     steady-state kernels; mesh cut-off, seismogram down-sampling, non-static compilation.\newline

 version 7.0.2, July 2019:
   Kazuto Ando, Etienne Bachmann, Hom Nath Gharti, Matthieu Lefebvre, Wenjie Lei, Dimitri Komatitsch, Andy Nowacki,
   Daniel Peter, Elliott Sales de Andrade, Malte Schirwon, James Smith, Kai Tao, Brice Videau, Victor (@MisterFruits).
     code improvements for adjoint reader for ASDF, CUDA and OpenCL, LDDRK, OpenMP and loop performance for compute forces,
     undo attenuation, BOAST kernels; support for full SH models, LIBXSMM, point force sources.\newline

 version 7.0.1, July 2015:
   Matthieu Lefebvre, Dimitri Komatitsch, Daniel Peter, Elliott Sales de Andrade, James Smith.
     code improvements (performance for noise simulations, ASDF provenance); updates examples.\newline

 version 7.0, January 2015:
   many developers.
     simultaneous MPI runs, ADIOS file I/O support, ASDF seismograms, new seismogram names, tomography tools,
     CUDA and OpenCL GPU support, CEM model support, updates AK135 model, binary topography files,
     fixes geocentric/geographic conversions, updates ellipticity and gravity factors, git versioning system.\newline

 version 6.0, April 2014:
   Daniel Peter (ETH Z\"urich, Switzerland), Dimitri Komatitsch and Zhinan Xie (CNRS / University of Marseille, France),
   Elliott Sales de Andrade (University of Toronto, Canada), and many others, in particular from Princeton University, USA.
     more flexible MPI implementation, GPU support, exact undoing of attenuation, LDDRK4-6 higher-order time scheme, etc...\newline

 version 5.1, February 2011:
   Dimitri Komatitsch, University of Toulouse, France and Ebru Bozdag, Princeton University, USA.
     non blocking MPI for much better scaling on large clusters;
     new convention for the name of seismograms, to conform to the IRIS standard;
     new directory structure.\newline

 version 5.0, February 2010:
   many developers.
     new moho mesh stretching honoring crust2.0 moho depths,
     new attenuation assignment, new SAC headers, new general crustal models,
     faster performance due to Deville routines and enhanced loop unrolling,
     slight changes in code structure.\newline

 version 4.0, February 2008:
   David Mich\'ea and Dimitri Komatitsch, University of Pau, France.
     first port to GPUs using CUDA, new doubling brick in the mesh, new perfectly load-balanced mesh,
     more flexible routines for mesh design, new inflated central cube
     with optimized shape, far fewer mesh files saved by the mesher,
     global arrays sorted to speed up the simulation, seismos can be
     written by the main process, one more doubling level at the bottom
     of the outer core if needed (off by default).\newline

 version 3.6, September 2006:
   Many people, many affiliations.
     adjoint and kernel calculations, fixed IASP91 model,
     added AK135 and 1066a, fixed topography/bathymetry routine,
     new attenuation routines, faster and better I/Os on very large
     systems, many small improvements and bug fixes, new "configure"
     script, new user's manual etc.\newline

 version 3.5, July 2004:
   Dimitri Komatitsch, Brian Savage and Jeroen Tromp, Caltech.
     any size of chunk, 3D attenuation, case of two chunks,
     more precise topography/bathymetry model, new Par\_file structure.\newline

 version 3.4, August 2003:
   Dimitri Komatitsch and Jeroen Tromp, Caltech.
     merged global and regional codes, no iterations in fluid, better movies.\newline

 version 3.3, September 2002:
   Dimitri Komatitsch, Caltech.
     flexible mesh doubling in outer core, inlined code, OpenDX support.\newline

 version 3.2, July 2002:
   Jeroen Tromp, Caltech.
     multiple sources and flexible PREM reading.\newline

 version 3.1, June 2002:
   Dimitri Komatitsch, Caltech.
     vectorized loops in solver and merged central cube.\newline

 version 3.0, May 2002:
   Dimitri Komatitsch and Jeroen Tromp, Caltech.
     ported to SGI and Compaq, double precision solver, more general anisotropy.\newline

 version 2.3, August 2001:
   Dimitri Komatitsch and Jeroen Tromp, Caltech.
     gravity, rotation, oceans and 3-D models.\newline

 version 2.2, March 2001:
   Dimitri Komatitsch and Jeroen Tromp, Caltech, USA.
     final MPI package.\newline

 version 2.0, January 2000:
   Dimitri Komatitsch, Harvard, USA.
     MPI code for the globe.\newline

 version 1.0, June 1999:
   Dimitri Komatitsch, UNAM, Mexico.
     first MPI code for a chunk.\newline

 Jeroen Tromp and Dimitri Komatitsch, Harvard, USA, July 1998: first chunk solver using OpenMP on a Sun machine.\newline

 Dimitri Komatitsch, IPG Paris, France, December 1996: first 3-D solver for the CM-5 Connection Machine,
    parallelized on 128 processors using Connection Machine Fortran.\newline

